\documentclass[12pt]{article}

% === PAGE SIZE SETTINGS ===
\usepackage[
  paperwidth=20cm,
  paperheight=28.7cm,
  margin=1.5cm  % margins inside the rectangle (you can adjust)
]{geometry}

% === FONTS AND TOOLS ===
\usepackage{lipsum}       % For dummy text (remove later)
\usepackage{amssymb}
\usepackage{amsfonts}
\usepackage{gensymb}
\usepackage{tikz}         % For drawing rectangle
\usepackage{everypage}    % To repeat on every page
% === DRAW RECTANGLE ON EVERY PAGE ===
\AddEverypageHook{%
  \begin{tikzpicture}[remember picture, overlay]
    \draw[thick]
      ([xshift=1cm,yshift=1cm]current page.south west) rectangle
      ([xshift=-1cm,yshift=-1cm]current page.north east);
  \end{tikzpicture}
}
\begin{document}
\title{Group Theory and Abstract Algebra}
\author{Sudip Gautam }
\date{\today}

\section{Abstract Algebra and Group Theory}
\subsection{Calculation Rule}
There are two basic operations used to calculate the addition and product of numbers: addition ($+$) and multiplication ($\cdot$). 

\textbf{1.  i. Associative Law.}
For all \(a,b,c\in\mathbb{R}\),
\[
  (a + b) + c = a + (b + c),
  \quad
  (a \cdot b) \cdot c = a \cdot (b \cdot c).
\]
This means that, for any real numbers \(a\), \(b\), and \(c\), both addition (using \(+\)) and multiplication (using \(\cdot\)) are associative.

\medskip

\noindent
\textbf{Example.}
Let \(a=1\), \(b=2\), and \(c=3\). Then
\[
  (1 + 2) + 3 \,= \, 1 + (2 + 3) \, = \, 6,
  \quad
  (1 \, \cdot \, 2) \cdot \, 3 \,= \, 1 \, \cdot \,(2 \,\cdot \, 3) \, = \ 6.
\]

\noindent
Note that associativity concerns the grouping of operands; it does not imply that you can swap their order, that property is commutativity.

\textbf{ ii. Commutative Law.}
For all \(a,b\in\mathbb{R}\),
\[
  a + b = b + a,
  \quad
  a \cdot b = b \cdot a.
\]
This means that, for any real numbers \(a\) and \(b\), the order of the operands does not affect the result of addition or multiplication.

\medskip

\noindent
\textbf{Example.}
Let \(a=1\) and \(b=2\). Then
\[
  1 + 2 = 2 + 1 = 3,
  \quad
  1 \cdot 2 = 2 \cdot 1 = 2.
\]

\noindent
Note that commutativity is a distinct property from associativity: it allows you to swap the order of two operandsc (which mean numbers here), but says nothing about how you group three or more operands.  

\textbf{ iii. Distributive Law.}
For all \(a,b,c\in\mathbb{R}\),
\[
  c \cdot (a + b) = c \cdot a + c \cdot b.
\]
This means that, for any real numbers \(a\), \(b\), and \(c\), multiplication distributes over addition: you can multiply a sum by a factor either before or after performing the addition.

\medskip

\noindent
\textbf{Example.}
Let \(a=1\), \(b=2\), and \(c=3\). Then
\[
  3 \cdot (1 + 2) = 3 \cdot 1 + 3 \cdot 2 = 9.
\]

\noindent
Note that distributivity is a bridge between multiplication and addition; it is neither purely about reordering (commutativity) nor just regrouping (associativity).  

\textbf{2. Identity and Inverse Elements.}
For all \(a\in\mathbb{R}\),
\[
  a + 0 = a,
  \quad
  a \cdot 1 = a.
\]
This states that \(0\) and \(1\) are the identity elements for addition and multiplication, respectively.
\\Furthermore,
\[
  \forall\,a\in\mathbb{R}: \quad a + (-a) = 0,
  \quad
  \forall\,a\in\mathbb{R}\setminus\{0\}:\quad a \, \cdot \, a^{-1} = 1.
\]
This indicates that every real number has an additive inverse, and every nonzero real number has a multiplicative inverse.

\medskip

\noindent
\textbf{Example.}
Let \(a=5\). Then
\[
  5 + 0 = 5,
  \quad
  5 \cdot 1 = 5,
  \quad
  5 + (-5) = 0.
\]
Let \(a=2\). Then
\[
  2 \cdot 2^{-1} = 1.
\]

\noindent
Note that \(0\) has no multiplicative inverse, so \(a^{-1}\) exists only for \(a\neq 0\).
\\
Further examples: But if we do product of a with 0 or, of inverse of inverse of a is
\[
  0 \cdot a = 0,
  \quad
  -(-a) = a,
\]  
and one also obtains the \emph{cancellation law}:  
\[
  (ab = ac \;\wedge\; a \neq 0)\;\Longrightarrow\; b = c.
\]

\medskip

\noindent
\textbf{Introduction to Algebraic Structures.}  
In this section we introduce the most important algebraic structures.  \emph{Groups} and \emph{rings} arise in many contexts throughout mathematics.  Of particular interest are \emph{fields}.
\section*{Monoids and Groups}

\textbf{Definition (Monoid).}  
A \emph{monoid} is a triple \((M,\star,e)\) where  
\begin{itemize}
  \item \(M\) is a set,
  \item \(\star: M \times M \to M\) is a binary operation,
  \item \(e \in M\) is a distinguished element, called the \emph{identity},
\end{itemize}
such that the following axioms hold:
\begin{enumerate}
  \item \textbf{Associativity:}  
    \[
      \forall\,a,b,c\in M:\quad (a\star b)\star c \;=\; a\star(b\star c).
    \]
  \item \textbf{Identity:}  
    \[
      \forall\,a\in M:\quad e\star a = a = a\star e.
    \]
\end{enumerate}
If, in addition,  
\[
  \forall\,a,b\in M:\quad a\star b = b\star a,
\]
then the monoid is called \emph{commutative} (or \emph{abelian}).  Note that commutativity is \emph{not} part of the definition of a monoid but is an extra property.
\\
\textbf{Example of a Monoid.}  
Consider the set of natural numbers including zero, \(\mathbb{N}_0 = \{0,1,2,3,\dots\}\), together with the operation of addition:
\[
  +\colon \mathbb{N}_0 \times \mathbb{N}_0 \to \mathbb{N}_0,\quad
  (a,b)\mapsto a + b,
\]
and the identity element \(0\).  Then
\begin{itemize}
  \item \emph{Associativity:} For any \(a,b,c\in\mathbb{N}_0\), \quad
    \((a + b) + c = a + (b + c).\)
  \item \emph{Identity:} For any \(a\in\mathbb{N}_0\),\quad
    \(0 + a = a = a + 0.\)
\end{itemize}
Hence \(\bigl(\mathbb{N}_0,+,0\bigr)\) is a monoid: you can add any two natural numbers, the result stays a natural number, and “adding zero” does nothing.

\bigskip

\textbf{Definition (Group).}  
A \emph{group} is a monoid \((G,\star,e)\) satisfying the extra axiom of \emph{invertibility}:
\[
  \forall\,a\in G,\;\exists\,a^{-1}\in G:\quad a\star a^{-1} = e = a^{-1}\star a.
\]
Thus every element of a group has a two‐sided inverse.  A group is called \emph{abelian} if its underlying monoid is commutative.

\bigskip

These two structures—monoid and group—form the foundation for more elaborate algebraic systems (rings, fields, modules, …) by adding further operations or axioms.
\\
\textbf{Example of a Group.}  
Now take the set of all integers, \(\mathbb{Z} = \{\dots,-2,-1,0,1,2,\dots\}\), again with addition and identity \(0\).  In addition to the monoid axioms, every integer \(a\) has an \emph{inverse} \(-a\) such that
\[
  a + (-a) = 0 = (-a) + a.
\]
Thus \(\bigl(\mathbb{Z},+,0\bigr)\) is a group: you can both add and “undo” addition by subtracting, and all the axioms (associativity, identity, inverses) hold. 
\\
\textbf{Example (Familiar Examples of Monoids)}

Here are some monoids you should know:
\begin{itemize}
  \item \(\bigl(\mathbb{N}, +, 0\bigr),\ (\mathbb{Q}, +, 0),\ (\mathbb{Z}, +, 0),\ (\mathbb{R}, +, 0)\).  
    The natural, rational, integer and real numbers with ordinary addition and \(0\) as the identity.  
    \emph{Commutative.}

  \item \(\bigl(\mathbb{N}, \cdot, 1\bigr),\ (\mathbb{N}_{>0}, \cdot, 1),\ (\mathbb{Q}, \cdot, 1),\ (\mathbb{Z}, \cdot, 1),\ (\mathbb{R}, \cdot, 1)\).  
    The same sets with ordinary multiplication and \(1\) as the identity.  
    \emph{Commutative.}

  \item \(\bigl(\mathrm{Map}(X,X),\circ,\mathrm{id}_X\bigr)\).  
    All functions from a set \(X\) to itself, with composition \(\circ\) and the identity map \(\mathrm{id}_X\).  
    (Commutative only if  $|X|$ = \,1)

  \item \(\bigl(\mathrm{Mat}_{m\times n}(\mathbb{R}),\,+,\,0\bigr)\).  
    All \(m\times n\) real matrices with matrix‐addition and the zero matrix \(0\).  
    \emph{Commutative.}

  \item \(\bigl(\mathrm{Mat}_{n\times n}(\mathbb{R}),\,\cdot,\,I_n\bigr)\).  
    All \(n\times n\) real matrices with matrix‐multiplication and the identity matrix \(I_n\).  
    (Commutative only when \(n=1\).)

  \item \(\bigl(\{e\},\star,e\bigr)\).  
    The \emph{trivial monoid} consisting of a single element \(e\) satisfying \(e\star e = e\).  
    \emph{Commutative.}
\end{itemize}
\textbf{Defination of \emph{Fields} }
A \emph{field} is a set on which the operations of addition, subtraction, multiplication, and division are defined and satisfy the same axioms as they do on the rational numbers \(\mathbb{Q}\) and the real numbers \(\mathbb{R}\).  A field is thus a fundamental algebraic structure. These are structures in which one has all four basic arithmetic operations together with the usual rules (for instance, \(\mathbb{R}\) is a field).
\\
\textbf{Definition (Axiom).}  
An \emph{axiom} is a statement accepted as true without proof, serving as a foundational assumption from which further results are logically deduced.

\medskip

\noindent
In the theory of fields, the \emph{field axioms} are the following requirements on a set \(F\) with two operations \(+\) and \(\cdot\):
\begin{enumerate}
  \item \textbf{Associativity:} \((a + b) + c = a + (b + c)\), \quad \((a\cdot b)\cdot c = a\cdot(b\cdot c)\).
  \item \textbf{Commutativity:} \(a + b = b + a\), \quad \(a\cdot b = b\cdot a\).
  \item \textbf{Distributivity:} \(a\cdot(b+c) = a\cdot b + a\cdot c\).
  \item \textbf{Identity elements:} There exist \(0,1\in F\) such that \(a + 0 = a\) and \(a\cdot 1 = a\).
  \item \textbf{Inverses:} For every \(a\in F\), there is \(-a\in F\) with \(a + (-a)=0\); and for every \(a\neq0\) there is \(a^{-1}\in F\) with \(a\cdot a^{-1}=1\).
\end{enumerate}

We proceed much as one does in secondary school: first we introduce a single binary operation (e.g.\ addition on \(\mathbb{N}\) or \(\mathbb{R}\)), then learn how to \emph{undo} it (e.g.\ subtraction, which leads to \(\mathbb{Z}\)), before considering structures with two operations (“addition” and “multiplication”) that culminate in the definition of a field.  In this way, we build fields step by step from simpler structures.

\end{document}